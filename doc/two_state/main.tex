\documentclass[a4paper]{article}
\usepackage[dvipdfmx]{graphicx}
\usepackage{braket}
\usepackage{amsmath}

\makeatletter
\renewcommand{\theequation}{  
  \thesection.\arabic{equation} }
  \@addtoreset{equation}{section}
\makeatother

\def\vector#1{\mbox{\boldmath $#1$}}
\def\braces#1{\left( #1 \right)}
\def\bracem#1{\left\{ #1 \right\}}
\def\braceb#1{\left[ #1 \right]}
\def\lefts{\left(}
\def\rights{\right)}
\def\leftb{\left[}
\def\rightb{\right]}
\def\leftm{\left\{}
\def\rightm{\right\}}
\newcommand{\pd}[2]{\frac{\partial #1}{\partial #2}}
\newcommand{\pdd}[3]{\frac{\partial^2 #1}{\partial #2 \partial #3}}
\begin{document}
\title{Analytic solution of two level dynamics}
\author{Rei Matsuzaki}
\maketitle

\section{Introduction}
In this documents, I derive the analytic solution for dynamics in two-level sytems.
I used atomic unit in the formulation.

\section{Two level model}
Consider a system with Hamiltonian $H_0$ having only two eigenstates, $\psi_a$ and $\psi_b$,
with eigen energies $E_a$ and $E_b$. Define $\omega_0=E_b-E_a$. The most general wave function for this system may be
written as
\begin{eqnarray}
  \Psi(t) = a(t)e^{-iE_at}\psi_a + b(t)e^{-iE_bt}\psi_b.
\end{eqnarray}
If the system is coupled via matrix element $-x/2$, the coefficient $a(t)$ and $b(t)$
obey the coupled equation
\begin{eqnarray}
  i \frac{d}{dt}
  \begin{pmatrix}
    a(t) e^{-iE_at} \\
    b(t) e^{-iE_bt}
  \end{pmatrix}
  =  
  \begin{pmatrix}
    E_a          & -\frac{x}{2} \\
    -\frac{x}{2} & E_b
  \end{pmatrix}
  \begin{pmatrix}
    a(t)e^{-iE_at} \\
    b(t)e^{-iE_bt}
  \end{pmatrix}
\end{eqnarray}
This leads to a system of differential equation for the expansion coefficient
\begin{eqnarray}
  \dot{a}(t) &=& \frac{ix}{2}b(t)e^{-i\omega_0t}  \label{eq:dota} \\
  \dot{b}(t) &=& \frac{ix}{2}a(t)e^{+i\omega_0t}. \label{eq:dotb}
\end{eqnarray}
By taking another time derivative of Eq. (\ref{eq:dota}) and substituting for $b$ and $\dot{b}$
we obtain the second order differential equation
\begin{eqnarray}
  \ddot{a}(t)
  &=& \frac{\omega_0x}{2}b(t)e^{-i\omega_0t} + \frac{ix}{2}\dot{b}(t)e^{-i\omega_0t} \\
  &=& -i\omega_0\dot{a}(t) - \frac{|x|^2}{4}a(t).
\end{eqnarray}
To solve this equation, we substitute a trial function $a(t)=\exp(i\lambda t)$ and obtain
\begin{eqnarray}
  \lambda^2 + \omega_0\lambda - \frac{x^2}{4} = 0.
\end{eqnarray}
The solution is
\begin{eqnarray}
  \lambda &=& \frac{\omega_0 \pm \Omega }{2}, \\
  \Omega  &=& \sqrt{\omega_0^2 + x^2}  
\end{eqnarray}
The general solution for $a(t)$ is
\begin{eqnarray}
  a(t) &=& e^{i\omega_0t/2} \braceb{ Ae^{+i\Omega t/2} +Be^{-i\Omega t/2} } \\
  b(t) &=& -\frac{2\Omega}{x} e^{-i\omega_0t/2}  \braceb{ Ae^{+i\Omega t/2} -Be^{-i\Omega t/2} } \\
\end{eqnarray}
Inf we choose as initial condition $a(0) = 0$ then the solution becomes
\begin{eqnarray}
  a(t) &=& e^{i\omega_0t/2} \braceb{\cos(\Omega t/2) - i\frac{\omega_0}{\Omega}\sin(\Omega t/2)} \\
  b(t) &=& e^{-i\omega_0t/2} \frac{xi}{\Omega} \sin(\Omega t/2)
\end{eqnarray}

\end{document}